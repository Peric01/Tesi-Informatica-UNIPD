\cleardoublepage
\phantomsection
\pdfbookmark{Compendio}{Compendio}
\begingroup
\let\clearpage\relax
\let\cleardoublepage\relax
\chapter*{Sommario}
Il presente elaborato descrive il lavoro svolto durante il periodo di \textit{stage}, della durata di circa
320 ore, dal laureando Marko Peric presso l'azienda Eurosystem S.p.A. dal 18/06/2024 al
13/08/2025.
L'elaborato illustra i processi, le metodologie e gli strumenti impiegati nella progettazione e nello sviluppo di un sistema \textit{honeypot} finalizzato al monitoraggio e alla rilevazione di attività malevole in rete. Il sistema si basa sull'installazione di servizi volutamente vulnerabili su un server Linux Debian, con lo scopo di attirare potenziali attaccanti e analizzarne i comportamenti. I \textit{log} generati dai servizi vengono successivamente raccolti, elaborati e trasferiti in InfluxDB per la loro conservazione, e resi disponibili tramite Grafana per consentire un'analisi e una visualizzazione efficace delle informazioni tramite \textit{dashboard}.
\section*{Struttura dell'elaborato}
In questa sezione viene presentata la struttura dell'elaborato, illustrando brevemente il contenuto di ciascun capitolo:
\begin{description}
    \item[{\hyperref[chap:]{Il primo capitolo}}] approfondisce il contesto aziendale in cui si è svolto lo \textit{stage}, fornendo una panoramica completa dell'azienda, dei suoi prodotti e servizi nel settore della \textit{cybersecurity}, degli strumenti tecnologici utilizzati e dell'organizzazione del \textit{team} di sicurezza informatica, analizzando inoltre l'approccio dell'azienda verso l'innovazione tecnologica;

    \item[{\hyperref[chap:]{Il secondo capitolo}}] identifica e analizza il problema alla base del progetto, definendo gli obiettivi specifici dello sviluppo del sistema \textit{honeypot}, i vincoli operativi e tecnici che hanno influenzato il lavoro, le prospettive di sviluppo futuro e le motivazioni personali che hanno guidato la scelta di questo percorso di \textit{stage};

    \item[{\hyperref[chap:]{Il terzo capitolo}}] documenta l'intero processo di sviluppo del sistema \textit{honeypot}, dall'analisi dei requisiti e dei casi d'uso alla progettazione dell'architettura, dalla fase di codifica alle attività di verifica e validazione, presentando il prodotto finale e valutando il grado di conformità rispetto ai requisiti iniziali;
    
    \item[{\hyperref[chap:]{Il quarto capitolo}}] presenta una valutazione retrospettiva dell'esperienza formativa, analizzando il raggiungimento degli obiettivi prefissati, le conoscenze tecniche acquisite nel campo della sicurezza informatica e le competenze professionali sviluppate durante il periodo di tirocinio presso l'azienda.
\end{description}
\section*{Criteri tipografici adottati}
Riguardo la stesura del testo, relativamente al elaborato sono state adottate le seguenti convenzioni tipografiche:
\begin{itemize}
	\item gli acronimi e le abbreviazioni sono raccolti in un'apposita sezione dedicata, denominata \hyperref[acronimi]{Elenco degli acronimi} e sono segnati con la seguente nomenclatura \gls{es};
    \item i termini ambigui o di uso non comune, invece, sono definiti all'interno del \hyperref[glossario]{Glossario} e sono segnati con la seguente nomenclatura \gls{esempio};
	\item i termini in lingua straniera o facenti parti del gergo tecnico sono evidenziati con il carattere \textit{corsivo}.
\end{itemize}
\endgroup
\vfill
