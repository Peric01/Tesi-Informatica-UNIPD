% Load variables
\newcommand{\myUni}{Università degli Studi di Padova}
\newcommand{\myDepartment}{Dipartimento di Matematica ``Tullio Levi-Civita''}
\newcommand{\myFaculty}{Corso di Laurea in Informatica}
\newcommand{\myTitle}{Un sistema Honeypot per l'analisi di attacchi informatici in ambiente controllato}
\newcommand{\myDegree}{Tesi di Laurea Triennale}
\newcommand{\profTitle}{Prof.}
\newcommand{\myProf}{Vardanega Tullio}
\newcommand{\graduateTitle}{Laureando}
\newcommand{\myName}{Marko Peric}
\newcommand{\myStudentID}{2011067}
\newcommand{\myAA}{2024-2025}
\newcommand{\myLocation}{Padova}
\newcommand{\myTime}{Agosto 2025}
% Acronyms
\newacronym{tsa}{TSA}{Termine solo acronimo}

% Glossary

\newglossaryentry{TermineSenzaAcronimo}{
    name={Nome del termine},
    sort=termine senza acronimo,
    description={Descrizione}
}
\PassOptionsToPackage{dvipsnames}{xcolor} % colori PDF/A

\usepackage{colorprofiles}
% PDF/A
% validate in https://www.pdf-online.com/osa/validate.aspx
\usepackage[a-1a,mathxmp]{pdfx}[2018/12/22]
\usepackage[T1]{fontenc}
\usepackage[utf8]{inputenc}
\usepackage[italian]{babel}
\usepackage{bookmark}
\usepackage{caption}
\usepackage{comment}
\usepackage{chngpage, calc} % centra il frontespizio
\usepackage{emptypage} % pagine vuote senza testatina e piede di pagina
\usepackage{epigraph} % per epigrafi
\usepackage{indentfirst} % rientra il primo paragrafo di ogni sezione
\usepackage{graphicx} % immagini
\usepackage[pdfa]{hyperref} % collegamenti ipertestuali
\usepackage{mparhack,relsize}  % finezze tipografiche
\usepackage{nameref} % visualizza nome dei riferimenti
\usepackage[font=small]{quoting} % citazioni
\usepackage{subfig} % sottofigure, sottotabelle
\usepackage[italian]{varioref} % riferimenti completi della pagina
\usepackage{longtable} % tabelle su più pagine
\usepackage[toc, acronym, automake]{glossaries}
\usepackage[backend=biber, style=verbose-ibid, hyperref, backref]{biblatex}
\usepackage{lmodern}
\usepackage[top=2.75cm, bottom=2.75cm, right=3cm, left=3.75cm]{geometry} % 1in+17pt+0.6cm
\usepackage{fancyhdr}
\usepackage{lipsum}
\usepackage{setspace}
\usepackage{titlesec}
\usepackage[cachedir=minted-caches]{minted} % https://it.overleaf.com/learn/latex/Code_Highlighting_with_minted
\usepackage{xcolor}
\usepackage{csquotes} % gestisce automaticamente i caratteri (")
\usepackage{etoolbox}
\usepackage[bottom]{footmisc}
\usepackage{zref-totpages}

% Define custom colors
\definecolor{hyperColor}{HTML}{0B3EE3}
\definecolor{tableGray}{HTML}{F5F5F7}
\definecolor{veryPeri}{HTML}{6667ab}

% Set line height
\linespread{1.5}

% Custom hyphenation rules
\hyphenation {
    data-base
    al-go-rithms
    soft-ware
}

% Images path
\graphicspath{{img/}}

% Force page color, as some editors set a grayish color as default
\pagecolor{white}

% Better spacing for footnotes
\setlength{\skip\footins}{5mm}
\setlength{\footnotesep}{5mm}

\setlength{\headheight}{14.5pt}
\addtolength{\topmargin}{-2.45pt}

% Add a subscript G to a glossary entry
\newcommand{\glox}{\textsubscript{\textbf{\textit{G}}}}

% Improvements the paragraph command
\titleformat{\paragraph}
{\normalfont\normalsize\bfseries}{\theparagraph}{1em}{}
\titlespacing*{\paragraph}
{0pt}{3.25ex plus 1ex minus .2ex}{1.5ex plus .2ex}

% Define use case environment
\newcounter{usecasecounter} % define a counter
\setcounter{usecasecounter}{0} % set the counter to some initial value
% Parameters
% #1: ID
% #2: Nome
\newenvironment{usecase}[2]{
    \renewcommand{\theusecasecounter}{\usecasename #1}  % this is where the display of the counter is overwritten/modified
    \refstepcounter{usecasecounter} % increment counter
    \vspace{2em}
    \par \noindent % start new paragraph
    {\normalsize \textbf{\usecasename #1: #2}} % display the title before the content of the environment is displayed
    \vspace{.5em}
}{
    \medskip
}
\newcommand{\usecasename}{UC}
\newcommand{\usecaseactors}[1]{\textbf{\\Attori Principali:} #1}
\newcommand{\usecasepre}[1]{\textbf{\\Precondizioni:} #1}
\newcommand{\usecasedesc}[1]{\textbf{\\Descrizione:} #1}
\newcommand{\usecasepost}[1]{\textbf{\\Postcondizioni:} #1}
\newcommand{\usecasealt}[1]{\textbf{\\Scenario Alternativo:} #1}

% Define risks environment
\newcounter{riskcounter} % define a counter
\setcounter{riskcounter}{0} % set the counter to some initial value
% Parameters
% #1: Title
\newenvironment{risk}[1]{
    \refstepcounter{riskcounter} % increment counter
    \par \noindent % start new paragraph
    \textbf{\arabic{riskcounter}. #1} % display the title before the content of the environment is displayed
}{
    \par\medskip
}
\newcommand{\riskname}{Rischio}
\newcommand{\riskdescription}[1]{\textbf{\\Descrizione:} #1.}
\newcommand{\risksolution}[1]{\textbf{\\Soluzione:} #1.}

% Apply fancy styling to pages
\pagestyle{fancy}
\fancyhf{}
\fancyhead[L]{\leftmark} % Places Chapter N. Chatper title on the top left
\fancyfoot[C]{\thepage} % Page number in the center of the footer

% Adds a blank page while increasing the page number
\newcommand\blankpage{ 
\clearpage
    \begingroup
    \null
    \thispagestyle{empty}
    \hypersetup{pageanchor=false}
    \clearpage
\endgroup
}

% Adds a blank page while increasing the page number
\newcommand\blankpagewithnumber{ 
  \clearpage
  \thispagestyle{plain} % Use plain page style to keep the page number
  \null
  \clearpage
}

% Increase page numbering
\newcommand\increasepagenumbering{
    \addtocounter{page}{+1}
}

% Make glossaries and bibliography
\makeglossaries
% Redefine the format for the glossary entries to be italic
\renewcommand*{\glstextformat}[1]{\textit{#1}\glox}
%\glsaddall

\bibliography{references/bibliography}
\defbibheading{bibliography} {
    \cleardoublepage
    \phantomsection
    \addcontentsline{toc}{chapter}{\bibname}
    \chapter*{\bibname\markboth{\bibname}{\bibname}}
}

% Code blocks handling w/ table of codes
\makeatletter
\ifdefined\NR@chapter
  \expandafter\@firstoftwo
\else
  \expandafter\@secondoftwo
\fi{\patchcmd\NR@chapter}{\patchcmd\@chapter}
  {\addtocontents{lot}{\protect\addvspace{10\p@}}}
  {\addtocontents{lot}{\protect\addvspace{10\p@}}%
   \addtocontents{lol}{\protect\addvspace{10\p@}}}
  {}{}
\makeatother

\renewcommand\listingscaption{Codice}
\renewcommand\listoflistingscaption{Elenco dei codici sorgenti}
\counterwithin*{listing}{chapter}
\renewcommand\thelisting{\thechapter.\arabic{listing}}

% Set up hyperlinks
\hypersetup{
    colorlinks=true,
    linktocpage=true,
    pdfstartpage=1,
    pdfstartview=,
    breaklinks=true,
    pdfpagemode=UseNone,
    pageanchor=true,
    pdfpagemode=UseOutlines,
    plainpages=false,
    bookmarksnumbered,
    bookmarksopen=true,
    bookmarksopenlevel=1,
    hypertexnames=true,
    pdfhighlight=/O,
    allcolors = hyperColor
}

% Set up captions
\captionsetup{
    tableposition=top,
    figureposition=bottom,
    font=small,
    format=hang,
    labelfont=bf
}

% When draft mode is on, the hyperlinks are removed. Useful when printing the document. To enable/disable, uncomment/comment the command
% \hypersetup{draft}

% prevents cleaning up the cache at the end of the run (needed to keep the unused caches, generated by other editions)
% \makeatletter
% \renewcommand*{\minted@cleancache}{}
% \makeatother

% Break lines in code blocks whe using inputminted
\setminted{breaklines}