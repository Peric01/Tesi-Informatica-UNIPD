\chapter{Introduzione}
\label{chap:introduzione}
\section{L'azienda}
Paragrafo che conterrà la descrizione dell'azienda scelta in cui è stato effettuato lo stage (Eurosystem S.p.A.).
% Eurosystem S.p.A. è un'azienda informatica che si occupa di consulenza, prodotti e servizi IT. Dal 2022, fa parte del Gruppo internazionale Smart4Engineering, leader europeo nella trasformazione digitale.
% Presenta 8 sedi in Nord e Centro Italia (Treviso, Bergamo, Bologna, Udine, Ferrara, Modena, Milano, Firenze) supportando oltre 1300 clienti di piccola e media impresa.
% Ad oggi offre soluzioni nei settori delle Business Applications e Industria 4.0, delle Immersive Technologies, della Cyber Security e delle Tech Solutions \& Services.\\
% Il logo dell'azienda è riportato in figura \ref{fig:logo_eurosystem}
\begin{figure}[H]
    \centering
    \includegraphics[alt={Testo alternativo dell'immagine}, width=0.4\columnwidth]{img/logo_azienda.png}
    \caption{Logo di Eurosystem S.p.A.}
    \label{fig:logo_eurosystem}
\end{figure} 

\section{L'idea}
Paragrafo che conterrà la descrizione dell'idea di stage proposta dall'azienda e come questa sia stata implementata in maniera sintetica.
%Lo stage proposto consiste nell'implementare un sistema honeypot per la rilevazione delle minacce informatiche. Il sistema sarà progettato per simulare vulnerabilità e attirare attacchi, consentendo così di analizzare le tecniche utilizzate dagli aggressori e migliorare le difese informatiche.

\section{Organizzazione del testo}
In questa sezione viene presentata la struttura del documento, illustrando brevemente il contenuto di ciascun capitolo. Inoltre, vengono descritte le principali convenzioni adottate per la stesura, tra cui le modalità di scrittura degli acronimi, dei termini presenti nel glossario e dei termini in lingua straniera o tecnico-specialistica.
% \begin{description}
%     \item[{\hyperref[chap:processi-metodologie]{Il secondo capitolo}}] descrive ...
    
%     \item[{\hyperref[chap:descrizione-stage]{Il terzo capitolo}}] approfondisce ...
    
%     \item[{\hyperref[chap:analisi-requisiti]{Il quarto capitolo}}] approfondisce ...
    
%     \item[{\hyperref[chap:progettazione-codifica]{Il quinto capitolo}}] approfondisce ...
    
%     \item[{\hyperref[chap:verifica-validazione]{Il sesto capitolo}}] approfondisce ...
    
%     \item[{\hyperref[chap:conclusioni]{Nel settimo capitolo}}] descrive ...
% \end{description}

% Riguardo la stesura del testo, relativamente al documento sono state adottate le seguenti convenzioni tipografiche:
% \begin{itemize}
% 	\item gli acronimi, le abbreviazioni e i termini ambigui o di uso non comune menzionati vengono definiti nel glossario, situato alla fine del presente documento;
% 	\item per la prima occorrenza dei termini riportati nel glossario viene utilizzata la seguente nomenclatura: %\gls{apig};
% 	\item i termini in lingua straniera o facenti parti del gergo tecnico sono evidenziati con il carattere \textit{corsivo}.
% \end{itemize}

% \begin{listing}[H]
% \begin{minted}{c}
% #include <stdio.h>
% int main() {
%     print("Hello, world!");
%     return 0;
% }
% \end{minted}
% \caption{Example of code}
% \label{listing:a}
% \end{listing}

\newpage