\section{Strumenti \textit{software}}

Le tecnologie e gli strumenti utilizzati a supporto delle attività lavorative sono molteplici e si possono suddividere in due categorie principali: strumenti organizzativi e strumenti produttivi.

\subsection{Strumenti organizzativi}
\begin{itemize}
    \item \textbf{Microsoft Teams}: principale strumento di comunicazione interna, utilizzato per coordinare le attività, condividere documenti e partecipare a riunioni o chiamate quando necessario;
    \item \textbf{Microsoft Outlook}: utilizzato per la gestione della posta elettronica e dell'agenda. Oltre a pianificare incontri e mantenere un flusso di comunicazione costante con i colleghi, viene impiegato anche per lo scambio di comunicazioni con i clienti e per la ricezione di notifiche relative a incidenti di sicurezza provenienti dalle aziende, consentendo così di garantire una comunicazione diretta e tempestiva in caso di necessità;
    \item \textbf{GitHub}: utilizzato come piattaforma di controllo versione per la gestione del codice sorgente, il monitoraggio delle modifiche e la collaborazione su progetti software.
\end{itemize}

\subsection{Strumenti produttivi}
\begin{itemize}
    \item \textbf{LaTeX}: impiegato per la redazione della documentazione, grazie alle sue avanzate capacità di formattazione e gestione di contenuti tecnici e scientifici. Consente di realizzare documenti professionali, con ottima resa tipografica di formule, grafici, bibliografie e glossari. La separazione tra contenuto e presentazione rende il codice sorgente riutilizzabile e adatto a collaborazioni in ambito accademico e lavorativo;
    \item \textbf{Python}: linguaggio di programmazione ampiamente utilizzato in ambito aziendale per attività di \textit{cybersecurity}, in particolare nell'automazione di processi, nell'analisi di log e nella creazione di strumenti di monitoraggio e rilevamento delle minacce. Grazie alle numerose librerie dedicate, consente di sviluppare rapidamente script per il testing della sicurezza, l'interazione con \gls{api} e la gestione di sistemi complessi;
    \item \textbf{Visual Studio Code (VSCode)}: ambiente di sviluppo leggero, modulare e altamente personalizzabile, adottato principalmente per la scrittura, il debugging e la manutenzione di script e progetti. Supporta numerosi linguaggi e strumenti tramite estensioni dedicate. Pur rappresentando l'IDE più diffuso tra i dipendenti, non è imposto: ciascun utente può scegliere soluzioni alternative in base alle proprie esigenze operative e preferenze personali;
    \item \textbf{Docker}: tecnologia di virtualizzazione basata su \textit{\gls{container}}, adottata per garantire ambienti isolati, scalabili e facilmente replicabili. Permette di standardizzare le applicazioni, includendo tutte le dipendenze necessarie per l'esecuzione. Grazie alla leggerezza e portabilità, semplifica lo sviluppo, il testing e la distribuzione, riducendo i problemi di compatibilità tra sistemi.
\end{itemize}
