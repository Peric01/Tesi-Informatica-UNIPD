\section{Strumenti \textit{software}}  
Le tecnologie e gli strumenti software a supporto delle attività lavorative si articolano principalmente in due grandi categorie: strumenti organizzativi e strumenti produttivi.  
\subsection{Strumenti organizzativi}  
Per quanto riguarda gli strumenti organizzativi, l'azienda fa ampio uso di \textbf{\textit{Microsoft Teams}}, che rappresenta il principale canale di comunicazione interna. Questo strumento consente di coordinare le attività quotidiane, condividere documenti in modo rapido ed efficace e partecipare a riunioni o chiamate di lavoro quando necessario.\\\\
Un ruolo altrettanto centrale è ricoperto da \textbf{\textit{Microsoft Outlook}}, utilizzato non solo per la gestione della posta elettronica e dell'agenda, ma anche per la pianificazione di incontri e il mantenimento di un flusso di comunicazione costante, sia con i colleghi che con i clienti. Outlook svolge inoltre una funzione cruciale nella ricezione di notifiche riguardanti incidenti di sicurezza provenienti da aziende esterne, garantendo così risposte tempestive in caso di necessità.\\\\
A completare questo insieme di strumenti organizzativi vi è \textbf{\textit{GitHub}}, adottato come piattaforma di controllo versione. Questo strumento consente la gestione strutturata del codice sorgente, il tracciamento delle modifiche e la collaborazione efficiente tra più sviluppatori.  
\subsection{Strumenti produttivi}  
Sul versante degli strumenti produttivi, un ruolo rilevante è svolto da \textbf{\textit{LaTeX}}, scelto per la redazione di documentazione tecnica e scientifica grazie alle sue elevate capacità di formattazione. La possibilità di gestire con precisione formule matematiche, grafici, bibliografie e glossari permette di ottenere documenti di elevata qualità tipografica. Il modello di separazione tra contenuto e presentazione rende inoltre il codice sorgente facilmente riutilizzabile e particolarmente adatto a contesti di collaborazione accademica e professionale.\\\\  
In parallelo, l'azienda fa un ampio utilizzo di \textbf{\textit{Python}}, linguaggio di programmazione versatile e diffuso, applicato soprattutto nel settore della \textit{cybersecurity}. Le sue potenzialità vengono sfruttate per automatizzare processi, analizzare file di log e sviluppare strumenti di monitoraggio e rilevamento delle minacce, anche grazie alle numerose librerie specializzate che semplificano la creazione di script per il testing della sicurezza, l'interazione con \gls{api} e la gestione di sistemi complessi.\\\\  
A supporto dello sviluppo viene frequentemente adottato \textbf{\textit{Visual Studio Code (VSCode)}}, un ambiente di programmazione leggero, modulare e altamente personalizzabile. Questo strumento è impiegato principalmente per la scrittura, il debugging e la manutenzione del codice, offrendo un'ampia compatibilità con diversi linguaggi e la possibilità di integrare estensioni dedicate. Pur essendo l'IDE più diffuso tra i dipendenti, la sua adozione non è vincolante: ciascun lavoratore può infatti optare per soluzioni alternative in base alle proprie esigenze e preferenze operative.\\\\  
Infine, un ulteriore strumento di grande rilevanza è \textbf{\textit{Docker}}, una tecnologia di virtualizzazione basata sui {\textit{container}. Grazie a questo approccio, è possibile creare ambienti isolati, scalabili e facilmente replicabili, in cui le applicazioni vengono eseguite includendo tutte le dipendenze necessarie. La leggerezza e portabilità della tecnologia ne favoriscono l'uso in fase di sviluppo, testing e distribuzione, riducendo al minimo i problemi di compatibilità tra diversi sistemi.  
