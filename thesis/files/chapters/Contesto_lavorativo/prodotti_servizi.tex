\section{Prodotti e servizi offerti}
Eurosystem S.p.A. offre una gamma di prodotti e servizi informatici pensati per garantire la protezione delle infrastrutture IT aziendali. In particolare, il reparto di \textit{cybersecurity} si occupa delle seguenti attività principali:

\begin{itemize}
    \item \textbf{Monitoraggio e risposta agli incidenti}: tramite il \textit{Security Operation Center} (SOC) e i servizi di \textit{Managed Detection and Response} (MDR), viene assicurato un monitoraggio continuo delle reti aziendali, con l'obiettivo di rilevare anomalie e possibili minacce in tempo reale. Ogni incidente viene analizzato in profondità attraverso:
    \begin{itemize}
        \item raccolta e analisi dei log e dei dati di rete per ricostruire l'accaduto;
        \item identificazione dei vettori di attacco e delle vulnerabilità sfruttate;
        \item isolamento delle minacce e ripristino dei sistemi tramite soluzioni di \textit{backup} e \textit{disaster recovery};
        \item produzione di report dettagliati con analisi, azioni intraprese e raccomandazioni di miglioramento.
    \end{itemize}

    \item \textbf{Verifiche di sicurezza}: comprendono attività di \textit{Cyber Analysis Assessment}, tra cui:
    \begin{itemize}
        \item \textit{Vulnerability Management} e \textit{Penetration Test} per identificare e valutare i punti deboli delle infrastrutture IT;
        \item \textit{Phishing Assessment} per testare la resilienza degli utenti contro campagne fraudolente;
        \item \textit{Threat Intelligence} per raccogliere e analizzare informazioni utili a prevenire attacchi futuri.
    \end{itemize}

    \item \textbf{Analisi delle reti}: i servizi di analisi delle reti hanno l'obiettivo di garantire la sicurezza e l'integrità delle infrastrutture di rete. Le attività principali includono:
    \begin{itemize}
        \item monitoraggio del traffico e rilevamento di anomalie;
        \item prevenzione e gestione delle intrusioni;
        \item protezione delle reti industriali per assicurare continuità operativa.
    \end{itemize}

    \item \textbf{Gestione degli \textit{endpoint}}: riguarda la protezione dei dispositivi aziendali, sia interni che remoti, attraverso:
    \begin{itemize}
        \item monitoraggio costante dello stato di sicurezza;
        \item gestione degli aggiornamenti correttivi e politiche di sicurezza;
        \item controllo delle identità e degli accessi;
        \item rilevamento e risposta rapida alle minacce;
        \item redazione di \textit{report} e \gls{audit} di conformità.
    \end{itemize}


    \item \textbf{Gestione dei dati}: supporto alle aziende nella protezione, nell'organizzazione e nella conformità dei dati, in linea con le normative vigenti.

    \item \textbf{Gestione degli utenti e sensibilizzazione}: attività di istruzione dei clienti finalizzata a diffondere buone pratiche di sicurezza informatica, con particolare attenzione agli accessi, ai privilegi e ai comportamenti sicuri. Queste attività contribuiscono a ridurre i rischi legati al fattore umano.

    \item \textbf{Supporto in caso di criticità}: assistenza diretta e tempestiva ai clienti nella gestione di incidenti o sospette violazioni, con l'obiettivo di ridurre al minimo l'impatto sull'operatività aziendale.
\end{itemize}

Questo insieme di attività consente di identificare in modo proattivo le vulnerabilità dei sistemi informativi, predisporre adeguate misure di protezione e garantire un aggiornamento costante delle competenze necessarie ad affrontare l'evoluzione delle minacce informatiche.
