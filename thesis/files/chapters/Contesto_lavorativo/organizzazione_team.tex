\section{Organizzazione del \textit{team}}  

\subsection{\textit{Chief Information Security Officer} (CISO)}  
Il \textit{\gls{ciso}} è la figura responsabile della strategia di sicurezza informatica all'interno dell'azienda.
Supervisiona tutte le attività del reparto di \textit{cybersecurity} e ne garantisce la coerenza con gli obiettivi aziendali e con le normative vigenti.
Durante il periodo di tirocinio, questa figura ha ricoperto anche il ruolo di tutor aziendale, fornendo supporto e guida nelle varie attività.
Tra le sue principali responsabilità rientrano la definizione delle politiche e delle strategie di sicurezza, la supervisione delle attività operative del \textit{team}, il coordinamento tra i diversi ruoli del reparto e la verifica della conformità agli standard di settore.
A ciò si aggiunge un impegno costante nel supporto e nella formazione interna, volto a diffondere buone pratiche di sicurezza informatica tra i dipendenti.  

\subsection{\textit{Penetration tester}}  
Il \textit{penetration tester} ha l'obiettivo di individuare le vulnerabilità presenti nei sistemi informatici attraverso l'esecuzione di attacchi simulati in un contesto controllato.
Il suo lavoro si articola in più fasi, che comprendono la pianificazione e l'esecuzione di \textit{penetration test}, la simulazione di scenari di attacco realistici e l'analisi delle debolezze riscontrate.
Al termine delle attività, il professionista redige report tecnici dettagliati nei quali vengono descritte le criticità emerse, corredati da proposte di contromisure e raccomandazioni operative, così da supportare l'azienda nell'adozione di strategie efficaci di mitigazione del rischio.  

\subsection{\textit{Security analyst}}  
Il \textit{security analyst} si occupa del monitoraggio costante dei sistemi informativi e della gestione operativa degli eventi di sicurezza, con l'obiettivo di rilevare tempestivamente potenziali incidenti e coordinare una risposta adeguata.
Le sue attività comprendono il controllo delle infrastrutture tramite strumenti avanzati di analisi, il rilevamento e la classificazione degli avvisi di sicurezza, nonché l'esame dei \textit{log} e delle anomalie di rete.
In caso di incidenti, fornisce supporto operativo nelle attività di risposta e contribuisce alla definizione di procedure correttive.
Inoltre, elabora linee guida e buone pratiche da condividere con clienti e colleghi, contribuendo così a diffondere una maggiore consapevolezza in materia di sicurezza informatica.  
