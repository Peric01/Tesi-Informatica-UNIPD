\section{Organizzazione del team}
\subsection{\textit{Chief Information Security Officer} (CISO)}
Il \textit{\gls{ciso}} è la figura responsabile della strategia di sicurezza informatica all'interno dell'azienda. Supervisiona le attività del reparto di \textit{cybersecurity} e assicura che siano coerenti con gli obiettivi aziendali e le normative vigenti. Durante il tirocinio, questa figura ha svolto il ruolo di tutor aziendale.  
Le principali attività comprendono:
\begin{itemize}
    \item definizione delle politiche e delle strategie di sicurezza informatica;
    \item supervisione delle attività operative del team di \textit{cybersecurity};
    \item coordinamento tra i diversi ruoli del reparto;
    \item verifica della conformità alle normative e agli standard di settore;
    \item supporto e formazione interna sulle tematiche di sicurezza.
\end{itemize}
\subsection{\textit{Penetration tester}}
Il \textit{penetration tester} ha l'obiettivo di individuare le vulnerabilità nei sistemi informatici, simulando attacchi controllati per verificarne la resilienza.  
Le principali attività comprendono:
\begin{itemize}
    \item pianificazione ed esecuzione di \textit{penetration test};
    \item simulazione di scenari di attacco realistici;
    \item identificazione e analisi delle vulnerabilità riscontrate;
    \item redazione di report tecnici con indicazioni sulle criticità rilevate;
    \item proposta di contromisure per mitigare i rischi individuati.
\end{itemize}
\subsection{\textit{Security analyst}}
Il \textit{security analyst} si occupa del monitoraggio continuo e della gestione operativa degli eventi di sicurezza, con l'obiettivo di rilevare e rispondere a potenziali incidenti.  
Le principali attività comprendono:
\begin{itemize}
    \item monitoraggio dei sistemi e delle infrastrutture tramite strumenti di analisi;
    \item rilevamento e classificazione degli avvisi di sicurezza;
    \item analisi dei log e delle anomalie di rete;
    \item supporto nella gestione e risposta agli incidenti informatici;
    \item elaborazione di linee guida e buone pratiche di sicurezza per clienti e colleghi.
\end{itemize}
