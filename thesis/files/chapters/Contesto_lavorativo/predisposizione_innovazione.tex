\section{Predisposizione all'innovazione}

L'ambito della \textit{cybersecurity} è caratterizzato da un'evoluzione continua: nuove vulnerabilità, strumenti di attacco e metodologie di compromissione emergono con frequenza sempre maggiore. Per questo motivo, l'innovazione e l'aggiornamento costante costituiscono elementi fondamentali per ogni azienda che operi in questo settore. Eurosystem S.p.A. ha sviluppato un approccio che integra formazione, ricerca e sperimentazione, con l'obiettivo di anticipare i rischi e fornire soluzioni di difesa sempre più efficaci ai propri clienti.  
Un aspetto centrale è la formazione continua: nei momenti di minore operatività i \textit{team} si dedicano allo studio di nuove tematiche, alla valutazione di strumenti emergenti e all'adozione di metodologie innovative. In questo modo rimangono aggiornati sull'evoluzione delle minacce informatiche e rafforzano le proprie competenze tecniche.  
In parallelo, viene dato ampio spazio alla sperimentazione pratica di strumenti e tecniche di attacco. L'azienda valuta regolarmente soluzioni utilizzate dai \textit{penetration tester} durante le verifiche di sicurezza sui sistemi dei clienti. Tra questi strumenti rientrano dispositivi come:
\begin{itemize}
    \item \textbf{\textit{USB Rubber Ducky}}, che permette di eseguire comandi automatici sui dispositivi a cui viene collegata, simulando un attacco tramite periferica \gls{usb}. L'azienda impiega l'\textit{USB Rubber Ducky} per facilitare le attività di \textit{penetration testing} in loco presso le aziende clienti, consentendo di valutare la sicurezza delle postazioni di lavoro rispetto a questo tipo di minacce;\\
    \begin{figure}[H]
    \centering
    \includegraphics[alt={USB Rubber Ducky}, width=0.8\columnwidth]{img/usb-rubber-ducky_mk2_2000x.jpg}
    \caption{Dispositivo \textit{USB Rubber Ducky}.}
    Fonte: \url{https://shop.hak5.org}
    \label{fig:usb-rubber-ducky}
    \end{figure}
    \item \textbf{\textit{LAN Turtle}}, un adattatore di rete \gls{lan} che consente di instaurare accessi remoti nascosti e di analizzare il traffico direttamente dall'interno della rete aziendale. I \textit{penetration tester} utilizzano il \textit{LAN Turtle} durante le attività di \textit{penetration testing} per verificare l'esposizione delle infrastrutture a minacce derivanti dall'inserimento fisico di dispositivi malevoli;
    \begin{figure}[H]
    \centering
    \includegraphics[alt={LAN Turtle}, width=0.8\columnwidth]{img/lan-turtle_2000x.jpg}
    \caption{Dispositivo \textit{LAN Turtle}.}
    Fonte: \url{https://shop.hak5.org}
    \label{fig:lan-turtle}
    \end{figure}
\end{itemize}
L'adozione di tali strumenti non ha un fine puramente dimostrativo, ma rappresenta un'attività concreta di valutazione del livello di sicurezza dei clienti, in quanto permette di evidenziare vulnerabilità legate al fattore umano e alla protezione fisica delle postazioni di lavoro.  
Oltre agli strumenti hardware, l'azienda analizza e adotta nuove piattaforme e metodologie orientate sia all'offensiva sia alla difensiva. Tra queste si possono citare:
\begin{itemize}
    \item l'impiego di tecniche avanzate di \textit{Threat Intelligence}, volte a raccogliere e analizzare informazioni relative a campagne malevole, infrastrutture di attacco e nuove vulnerabilità sfruttate in scenari reali;
    \item l'aggiornamento continuo dei framework di \textit{penetration testing} e \textit{vulnerability management}, che consente di disporre di strumenti più precisi ed efficaci per identificare punti deboli e priorità di intervento;
    \item la sperimentazione di approcci innovativi al \textit{red teaming}, in cui si simulano operazioni complesse di attacco, con l'obiettivo di valutare non soltanto le difese tecnologiche ma anche le capacità organizzative e di risposta dei clienti.
\end{itemize}
In conclusione, la predisposizione all'innovazione non è intesa come un'attività separata, ma come parte integrante delle operazioni quotidiane. L'integrazione di formazione, sperimentazione e ricerca applicata consente al reparto di \textit{cybersecurity} di affrontare in maniera proattiva le sfide poste dalle minacce informatiche, mantenendo un equilibrio tra lo sviluppo delle competenze interne e l'offerta di servizi sempre più efficaci e aggiornati per i clienti.
