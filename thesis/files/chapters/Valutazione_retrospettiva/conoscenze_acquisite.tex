\section{Conoscenze acquisite}
Con lo svolgimento dello \textit{stage} ho avuto l'opportunità di apprendere nuove conoscenze e competenze che ritengo fondamentali per il mio sviluppo professionale e personale.
L'esperienza presso Eurosystem S.p.A. ha rappresentato un momento di crescita significativo, permettendomi di colmare alcune lacune formative del percorso universitario e di acquisire competenze pratiche direttamente applicabili nel mondo professionale
\subsection*{Competenze tecniche}
Una delle aree di maggiore apprendimento è stata la \textit{cybersecurity} applicata, settore che non avevo approcciato durante il percorso accademico. Ho acquisito competenze pratiche nella progettazione e implementazione di sistemi \textit{honeypot}, comprendendo le dinamiche degli attacchi informatici e le metodologie di difesa. La configurazione di servizi vulnerabili come\textit{ Cowrie, Dionaea, Apache} e \textit{OpenLDAP} mi ha permesso di comprendere concretamente le tipologie di vulnerabilità più comuni e le tecniche utilizzate dagli attaccanti per sfruttarle.
L'utilizzo dello \textit{stack TIG} ha rappresentato un'importante acquisizione in ambito \textit{data pipeline} e visualizzazione. Ho appreso come progettare e implementare sistemi \textit{ETL} scalabili, dalla raccolta automatizzata dei \textit{log} fino alla loro presentazione attraverso \textit{dashboard} interattive. Questa esperienza mi ha fornito competenze trasversali nella gestione di grandi volumi di dati e nella loro analisi in tempo reale.
La padronanza del protocollo \textit{MQTT} per la comunicazione tra componenti distribuite ha ampliato le mie conoscenze sui protocolli di messaggistica moderni, particolarmente rilevanti nell'ambito \gls{iot} e nei sistemi di monitoraggio in tempo reale. Ho inoltre approfondito l'uso di \textit{Docker} per la \textit{containerizzazione}, acquisendo competenze pratiche nella gestione di ambienti isolati e nella configurazione di architetture \textit{multi-container}.
Inoltre, l'applicazione pratica di \textit{design pattern} appresi durante il corso di Ingegneria del \textit{Software} ha consolidato la mia comprensione dell'architettura \textit{software}.
Ho implementato i \textit{pattern} nel contesto di un progetto reale, comprendendo concretamente i vantaggi in termini di modularità, scalabilità e manutenibilità del codice.
\subsection*{Capacità di \textit{problem solving}, autonomia e competenze trasversali}
Il lavoro da remoto ha sviluppato significativamente le mie capacità di autogestione e organizzazione. Ho imparato a pianificare autonomamente le attività, gestire le priorità e mantenere un dialogo costante con il tutor aziendale per garantire l'allineamento con gli obiettivi.
Le difficoltà tecniche incontrate, in particolare nella gestione delle espressioni regolari per il \textit{parsing} dei \textit{log} e nella configurazione dell'ambiente \textit{Docker}, mi hanno insegnato l'importanza di un approccio metodico al \textit{debugging} e della documentazione accurata dei problemi riscontrati. Queste esperienze hanno rafforzato la mia capacità di analizzare problemi complessi, scomporli in sotto-problemi più gestibili e trovare soluzioni efficaci.
L'esperienza ha inoltre rafforzato le mie competenze comunicative e di documentazione tecnica, attraverso la redazione di documentazione dettagliata e la presentazione dei risultati al \textit{team} aziendale.