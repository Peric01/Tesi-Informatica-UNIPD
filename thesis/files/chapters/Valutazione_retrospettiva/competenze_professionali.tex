\section{Competenze professionali acquisite}
Il quarto capitolo non si limita a elencare le nuove conoscenze sviluppate, ma mette in evidenza come lo \textit{stage} abbia consolidato e trasformato le competenze già presenti, segnando un passaggio importante dall'ambito accademico a quello professionale. 
\subsection*{Applicazione pratica delle conoscenze accademiche}
Il percorso universitario mi aveva fornito solide basi teoriche, in particolare nell'ambito della programmazione e dell'ingegneria del \textit{software}, che ho potuto finalmente applicare in un contesto reale. La possibilità di lavorare su un progetto completo mi ha permesso di tradurre in pratica concetti affrontati in aula, come la programmazione a oggetti, le metodologie di progettazione e la gestione di un progetto. Anche la capacità di ricerca autonoma, maturata durante gli anni di studio, si è rivelata un elemento essenziale per affrontare tecnologie e strumenti che conoscevo solo superficialmente e che richiedevano un rapido approfondimento.
L'esperienza in azienda ha rappresentato un ponte concreto tra teoria e pratica, consentendomi non solo di applicare le conoscenze già acquisite, ma anche di sviluppare competenze di natura organizzativa e relazionale. 
\subsection*{Crescita professionale e organizzativa}
Il lavoro a stretto contatto con professionisti esperti e con il mio tutor aziendale ha favorito una crescita significativa in termini di collaborazione, comunicazione e gestione delle attività. Ho imparato a condividere i progressi, a coordinarmi in modo efficace con il resto del gruppo e a rispettare scadenze precise, sviluppando al tempo stesso una maggiore autonomia decisionale.
Un aspetto particolarmente rilevante è il rafforzamento della mia capacità di gestire contemporaneamente compiti diversi: sviluppo del codice, configurazione dell'infrastruttura e redazione della documentazione. Questo esercizio di \textit{multitasking}, inizialmente complesso, si è rivelato fondamentale per affrontare la complessità tipica dei progetti aziendali. Allo stesso modo, la pianificazione delle attività e la capacità di rispettare tempi e priorità hanno rappresentato una sfida importante, che mi ha aiutato a maturare competenze organizzative precedentemente sviluppate solo in parte durante il percorso accademico.

In sintesi, lo \textit{stage} ha contribuito a trasformare conoscenze accademiche in strumenti concreti di lavoro, consolidando la mia professionalità e preparandomi a gestire con maggiore sicurezza le dinamiche proprie del mondo aziendale.
