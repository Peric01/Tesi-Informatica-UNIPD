\section{Obiettivi}
Gli obiettivi che sono suddivisi in minimi, obbligatori, desiderabili o facoltativi e sono elencati di seguito.
Utilizziamo la seguente nomenclatura per suddividere questi ultimi nelle quattro tipologie:
\begin{itemize}
    \item \textbf{M}: sono gli obiettivi vincolanti e necessari per il completamento del progetto; 
    \item \textbf{O}: sono gli obiettivi primari richiesti per il progetto;
    \item \textbf{D}: sono gli obiettivi non vincolanti o strettamente necessari, ma che portano valore aggiunto al prodotto;
    \item \textbf{F}: sono gli obiettivi che portano valore aggiunto al prodotto in modo non indispensabile.
\end{itemize}
Ogni obiettivo è contrassegnato con un numero e la categoria corrispondente, al fine di garantire una classificazione chiara e ordinata.
\begin{center}
\begin{longtable}{|p{0.2\textwidth}|p{0.75\textwidth}|}
\hline
\multicolumn{1}{|c|}{\textbf{Identificativo}} & \multicolumn{1}{c|}{\textbf{Obiettivo}}\\ 
\hline 
\endfirsthead
\multicolumn{2}{c}{{\bfseries \tablename\ \thetable{} -- Continuo della tabella}}\\
\hline
\multicolumn{1}{|c|}{\textbf{Identificativo}} & \multicolumn{1}{c|}{\textbf{Obiettivo}}\\ \hline 
\endhead
\hline
\multicolumn{2}{|r|}{{Continua nella prossima pagina...}}\\
\hline
\endfoot
\endlastfoot 

\multicolumn{2}{|c|}{\textbf{Obiettivi Minimi}} \\ \hline
M01 & Il codice del progetto deve essere scritto in \textit{Python}, deve avere struttura modulare e commenti esplicativi. \\ \hline
M02 & Il progetto deve presentare configurazioni leggibili, versionate e testate. \\ \hline
M03 & Gli \textit{script} devono essere riutilizzabili e gli \textit{input} devono essere parametrizzabili. \\ \hline
M04 & Il progetto deve avere una documentazione coerente con quanto implementato, scritta in forma tecnica, chiara e verificabile. \\ \hline

\multicolumn{2}{|c|}{\textbf{Obiettivi Obbligatori}} \\ \hline
O01 & Deve essere installata e configurata una macchina virtuale isolata per ambienti \textit{honeypot}. \\ \hline
O02 & L'\textit{honeypot} prodotto deve essere realistico con configurazione di servizi vulnerabili (\textit{Apache}, \gls{ftp}\glsadd{ftp_def}, \gls{smb}\glsadd{smb_def}, \gls{mqtt}, ecc.). \\ \hline
O03 & Devono essere sviluppati servizi \textit{dummy} con \textit{netcat} o strumenti equivalenti per l'ascolto dei pacchetti. \\ \hline
O04 & I \textit{log} devono essere raccolti e centralizzati tramite \textit{Python} e \textit{pipeline} \gls{tig}. \\ \hline
O05 & La raccolta dei \textit{log} deve essere automatizzata mediante \textit{script} \textit{Python}/\textit{Bash}. \\ \hline

\multicolumn{2}{|c|}{\textbf{Obiettivi Desiderabili}} \\ \hline
D01 & Deve essere effettuata un'analisi tecnica degli attacchi ricevuti con relativa correlazione dei dati. \\ \hline
D02 & Devono essere integrati strumenti e tecniche di \textit{Threat Intelligence} e \gls{osint}\glsadd{osint_def} per finalità di attribuzione. \\ \hline
D03 & Devono essere simulati attacchi controllati con strumenti noti e devono essere raccolti i relativi \textit{output}. \\ \hline
D04 & Devono essere applicate tecniche di base di \textit{data analysis} per il riconoscimento di \textit{pattern}. \\ \hline
D05 & Deve essere realizzata una \textit{dashboard} con grafici, \gls{ioc}\glsadd{ioc_def} e \textit{timeline} degli eventi. \\ \hline

\multicolumn{2}{|c|}{\textbf{Obiettivi Facoltativi}} \\ \hline
F01 & Devono essere implementati controlli avanzati di \textit{audit} con \textit{auditd} e \textit{logging} \textit{Bash} persistente. \\ \hline
F02 & Deve essere sperimentata la distribuzione dei dati verso \gls{siem}\glsadd{siem_def} \textit{open source}. \\ \hline
F03 & Devono essere introdotti strumenti di \gls{ai}/\gls{ml} per l'identificazione automatica di anomalie. \\ \hline

\caption{Elenco degli obiettivi suddivisi per categoria.}
\label{tab:obiettivi}
\end{longtable}
\end{center}
