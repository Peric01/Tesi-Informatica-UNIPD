\section{Motivazioni personali}
La scelta di questo progetto di tirocinio è stata fortemente influenzata dal desiderio di avvicinarmi al mondo della \textit{cybersecurity}, un ambito che negli ultimi anni ha assunto un ruolo sempre più centrale sia in ambito accademico che professionale. Durante il percorso universitario non ho avuto molte occasioni di affrontare questo tema in maniera pratica, e lo sviluppo di un sistema di \textit{honeypot} ha rappresentato per me un'occasione per approfondire un settore che considero strategico per il futuro delle tecnologie digitali.
La \textit{cybersecurity} non riguarda soltanto la protezione dei sistemi informatici, ma ha un impatto diretto sulla vita delle persone, delle imprese e delle istituzioni. Spesso, quando si parla di informatica, ci si concentra su aspetti come lo sviluppo di nuove applicazioni o l'ottimizzazione delle prestazioni, trascurando la componente della sicurezza. Tuttavia, gli attacchi informatici sono oggi una delle principali minacce globali, e questo rende evidente l'importanza di acquisire competenze specifiche in questo settore.  
Un altro elemento che ha influito sulla mia scelta è stato il carattere applicativo del progetto, che non si limitava a un esercizio teorico ma richiedeva lo sviluppo di un sistema reale, con vincoli tecnologici e obiettivi concreti. 
Un'ulteriore motivazione è stata la possibilità di sperimentare tecnologie che non avevo avuto modo di approfondire nel corso degli studi, come lo \textit{stack} \textit{TIG} e il protocollo \textit{MQTT}. Confrontarmi con strumenti utilizzati in contesti professionali rappresentava una sfida e al tempo stesso un'opportunità di crescita che avrebbe arricchito il mio percorso formativo.  
Lo svolgimento del tirocinio in modalità da remoto costituiva un'occasione importante per sviluppare autonomia e capacità di organizzazione. Gestire in modo indipendente il progetto e le attività quotidiane mi ha permesso di migliorare le mie competenze di pianificazione e di responsabilità, qualità che considero fondamentali per affrontare con successo future esperienze lavorative e accademiche.  
Le motivazioni che mi hanno portato alla scelta di questo progetto di tirocinio si sono tradotte in una serie di obiettivi personali.\\
In particolare, gli obiettivi prefissati erano i seguenti:
\begin{center}
\begin{longtable}{|p{0.15\textwidth}|p{0.8\textwidth}|}
\hline
\multicolumn{1}{|c|}{\textbf{ID}} & \multicolumn{1}{c|}{\textbf{Obiettivo personale}}\\ 
\hline 
\endfirsthead
\multicolumn{2}{c}{{\bfseries \tablename\ \thetable{} -- Continuo della tabella}}\\
\hline
\multicolumn{1}{|c|}{\textbf{ID}} & \multicolumn{1}{c|}{\textbf{Obiettivo personale}}\\ \hline 
\endhead
\hline
\multicolumn{2}{|r|}{{Continua nella prossima pagina...}}\\
\hline
\endfoot
\endlastfoot 

OP1 & Progettare e realizzare un sistema di \textit{honeypot} completo, partendo dall'analisi dei requisiti fino al collaudo finale, al fine di acquisire esperienza nello sviluppo di un progetto dall'inizio alla fine. \\ \hline
OP2 & Sviluppare un prodotto \textit{software} adottando i \textit{design pattern} e i principi architetturali appresi nel corso di Ingegneria del \textit{Software}, con particolare attenzione a modularità, scalabilità e manutenibilità del codice. \\ \hline
OP3 & Sviluppare competenze pratiche in ambito \textit{cybersecurity}, configurando e monitorando differenti tipologie di servizi vulnerabili da utilizzare all'interno del sistema di \textit{honeypot}. \\ \hline
OP4 & Acquisire competenze pratiche nell'utilizzo di tecnologie professionali come lo \textit{stack} \textit{TIG} e il protocollo \textit{MQTT}, comprendendone i principi di funzionamento e sperimentandone l'integrazione in una pipeline di raccolta e visualizzazione dati. \\ \hline
OP5 & Approfondire le competenze di analisi dei dati raccolti, individuando pattern ricorrenti negli attacchi e distinguendo tra traffico lecito e malevolo mediante metriche descrittive e dashboard interattive. \\ \hline

\caption{Obiettivi personali del tirocinio.}
\label{tab:obiettivi-personali}
\end{longtable}
\end{center}
Questi obiettivi riflettono le motivazioni che mi hanno spinto a scegliere questo progetto di tirocinio e rappresentano le competenze che intendevo sviluppare durante questa esperienza. Raggiungerli avrebbe significato non solo completare con successo il tirocinio, ma anche arricchire il mio bagaglio di conoscenze e abilità, preparandomi al meglio per le sfide future nel campo dell'informatica e della \textit{cybersecurity}.