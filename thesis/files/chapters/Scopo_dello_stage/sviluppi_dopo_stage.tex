\section{Prospettive future}
Sono presenti diverse possibilità di evoluzione, finalizzate a incrementarne le funzionalità e l'efficacia del progetto. 
Una possibile estensione riguarda l'aggiunta di ulteriori servizi vulnerabili nel sistema \textit{honeypot}, con l'obiettivo di renderlo più realistico e in grado di raccogliere un numero maggiore di dati sulle tecniche di attacco.  
Si potrebbe inoltre rendere il sistema esposto alla rete, rimuovendo alcune limitazioni imposte dai \textit{firewall}, in modo da osservare interazioni più eterogenee con l'ambiente esterno e analizzare comportamenti degli attaccanti più complessi.  
Un'ulteriore evoluzione riguarda l'introduzione di meccanismi automatici per l'identificazione delle anomalie, sfruttando strumenti di \textit{AI} e \textit{ML}. Questo permetterebbe di automatizzare il rilevamento delle minacce e di ridurre i tempi di risposta agli attacchi.  
È inoltre possibile estendere il sistema di \textit{logging} per supportare più lingue, migliorando l'usabilità e la fruibilità dei dati da parte di \textit{team} internazionali o utenti non madrelingua.  
Infine, il progetto potrebbe essere reso un pacchetto unico distribuibile ai clienti, semplificando l'installazione e la configurazione del sistema e favorendone l'adozione in contesti differenti.  
Queste evoluzioni rappresentano opportunità concrete per aumentare il valore del sistema e garantirne una maggiore flessibilità, automatizzazione e facilità d'uso.
