\section{Il ruolo dei tirocini in Eurosystem S.p.A.}
Eurosystem S.p.A. considera i tirocini un elemento strategico sia per la formazione dei giovani sia per l'innovazione aziendale nel settore della \textit{cybersecurity}.  
Essi permettono di avvicinarsi concretamente al mondo del lavoro e di contribuire ai progetti aziendali.  
L'azienda non interpreta il tirocinio soltanto come un'esperienza formativa, ma anche come un'opportunità per sperimentare nuove idee di progetto e per valutare soluzioni tecnologiche che, se ritenute efficaci, possono essere integrate nei servizi offerti ai clienti con l'obiettivo di rafforzarne la sicurezza informatica.  
I tirocinanti vengono accolti in un contesto lavorativo dinamico, caratterizzato da attività concrete e da un costante confronto con professionisti del settore.  
Fin dal loro inserimento, hanno la possibilità di applicare le conoscenze già acquisite e di arricchirle con competenze operative, maturate direttamente a contatto con problematiche reali della \textit{cybersecurity}.  
Questa modalità consente non solo di consolidare il proprio bagaglio tecnico, ma anche di sviluppare capacità trasversali come la collaborazione, la creatività e il \textit{problem solving}, indispensabili per affrontare scenari complessi e in continua evoluzione.  
Il tirocinio assume così un ruolo bidirezionale: da un lato favorisce la crescita formativa e professionale del tirocinante, dall'altro rappresenta per Eurosystem un'occasione per esplorare nuove prospettive progettuali e per individuare talenti da coinvolgere attivamente nelle sfide della sicurezza informatica.  